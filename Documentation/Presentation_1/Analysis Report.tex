\documentclass[a4paper, 12pt]{scrartcl} %A4Format, 12ptSchrift, "scrartclt" für Berichte
\usepackage[utf8]{inputenc} %utf8 mit Umlauten
\usepackage[ngerman]{babel} %deutsches Sprachpaket
\usepackage{mathptmx}
\usepackage[T1]{fontenc} %AusgabeFont
\usepackage{graphicx} %Einbinden von Grafiken
\usepackage{titletoc} %Formatierung von TableofContent
\usepackage{here}
\usepackage{underscore}
\usepackage[headsepline]{scrpage2}
\pagestyle{scrheadings}
\clearscrheadfoot
\ihead{Studium Generale}
\ohead{Schmidgall, Markus}
\ifoot{\footnotesize Hochschule Aalen}
\ofoot{\footnotesize Seite \pagemark}

\author{Markus Schmidgall}
\date{\today}

\begin{document}
\begin{titlepage}
	\centering
	\includegraphics[width=1\textwidth]{Logo.png}\par\vspace{3cm}
	{\huge\bfseries Zusammenfassung der Sozialpunkte nach der Richtlinie des Studium Generale an der Hochschule Aalen\par}
	\vspace{4cm}
	{\large Eingereicht von\par
	Markus Schmidgall \par
	Studiengang Software Engineering\par
	Matr.-Nr. 53592}
	\vfill

% Bottom of the page
	Aalen, \today\par
\end{titlepage}

\tableofcontents
\clearpage
\section{Zusammenfassung der besuchten Veranstaltungen}
\vspace{0.3cm}
Während meines Studiums habe ich an folgenden Vorträgen und Seminaren des Studium Generale
teilgenommen:
\vspace{0.7cm}
\begin{enumerate}
\item\textbf{Helfer AIM} \par
Termin: 29. Oktober 2014 \par
Workload: 6 Stunden
\item\textbf{Unternehmenskolloquium IBM} \par
Frank Koeble \par
Termin: 17. November 2015 \par
Workload: 5 Stunden
\item\textbf{MOSTflexiPL - Flexibel wie Knetmasse - Eine Programmiersprache ohne Korsett} \par
Prof. Dr. Christian Heinlein \par
Termin: 8. Dezember 2015 \par
Workload: 5 Stunden
\item\textbf{Vortragstechnik – Professionell Präsentieren in Studium und Beruf} \par
Termin: 16. November 2016 \par
Workload: 10 Stunden
\item\textbf{Externe ehrenamtliche Tätigkeit Projekt ''Studierende helfen Geflüchteten''} \par
Ernst Zwilling \par
Termin: 1. Oktober 2016 - 31. Februar 2017 \par
Workload: 28 Stunden
\item\textbf{Forensic Readiness: Intrusion Prevention from the Inside Out} \par
Dr. Jan Collie, Principal \& Senior, Forensic Investigator, Cambridge \par
Termin: 19. März 2018, 17:30 - 19:30 \par
Workload: 5 Stunden
\item\textbf{Erkenntnis: Was kann ich wissen?} \par
Prof. Dr. Karl Mertens, Universität Würzburg \par
Termin: 26. März 2018, 18:00 - 20:00  \par
Workload: 5 Stunden
\item\textbf{Cybersicherheit in der Digitalisierung} \par
Arne Schönbohm, Präsident des Bundesamtes für Sicherheit in der Informationstechnik\par
Termin:9. April 2018, 17:30 - 19:30 \par
Workload: 5 Stunden
\item\textbf{Ethik Cafe Zeug über Mensch} \par
Michael Wratschko, Optima nonwovens GmbH\par
Termin: 16. April 2018, 17:30 - 19:30  \par
Workload: 5 Stunden
\item\textbf{Ethik: Was soll ich tun?} \par
Prof. Dr. Karen Joisten, TU Kaiserslautern\par
Termin: 7. Mai 2018, 18:00 - 20:00  \par
Workload: 5 Stunden
\item\textbf{Safe performance with Total Care Tools} \par
Michael Wratschko, Optima nonwovens GmbH\par
Termin: 14. Mai 2018, 17:30 - 19:30  \par
Workload: 5 Stunden
\item\textbf{Live-Hacking: So brechen digitale Angreifer in ihre Systeme ein} \par \par
Termin: 4. Juni 2018, 17:30 - 19:30  \par
Workload: 5 Stunden
\item\textbf{Sinn: Was darf ich hoffen?} \par
Dr. Sebastian Schwenzfeuer, Universität Freiburg\par
Termin: 11. Juni 2018, 18:00 - 20:00  \par
Workload: 5 Stunden
\end{enumerate}
\clearpage
\section{Beschreibung der besuchten Veranstaltungen}
\begin{enumerate}
\item\textbf{Helfer AIM}\par

\vspace{0.5cm}
\item\textbf{Unternehmenskolloquium IBM}\par

\vspace{0.5cm}
\item\textbf{MOSTflexiPL - Flexibel wie Knetmasse - Eine Programmiersprache ohne Korsett}\par

\vspace{0.5cm}
\item\textbf{Vortragstechnik – Professionell Präsentieren in Studium und Beruf}\par

\vspace{0.5cm}
\item\textbf{Externe ehrenamtliche Tätigkeit Projekt ''Studierende helfen Geflüchteten''}\par

\vspace{0.5cm}
\item\textbf{Forensic Readiness: Intrusion Prevention from the Inside Out}\par
Da die Sicherheit informationstechnischer Systeme, und vor allem die Geheimhaltung persönlicher Daten in den vergangenen Jahren immer mehr an Bedeutung gewonnen hat, und dies auch eine zentrale Rolle im Beruf des Softwareingenieurs spielt, besuchte ich den Vortrag von Frau Dr. Jan Collie und auch alle weiterführenden Beiträge der Vortragsreihe bezüglich Cybersicherheit im Sommersemester 2018.\\Interessant war es zu erfahren, wie sicherheitstechnische Aspekte und Konzepte der Informatik durch Frau Collie und ihr Unternehmen in der Praxis umgesetzt werden. Des Weiteren wurde verdeutlicht, wo in Unternehmen mehrheitlich Schwachstellen im Sicherheitssystem auftauchen und wie diese, durch Schulung des richtigen Verhaltens und Handelns jedes einzelnen Mitarbeiters deutlich reduziert werden können.
\vspace{0.5cm}
\item\textbf{Erkenntnis: Was kann ich wissen?}\par\
Im Vortrag von Herrn Prof. Dr. Karl Mertens ging es um die Frage, was der Mensch, unabhängig seiner Erkenntnis und seines eigenen Bewusstseins wissen kann. Was sind die Voraussetzungen dafür, dass der Mensch überhaupt etwas wissen kann. Es wurde vor allem Immanuel Kants Sicht auf diese Fragestellung behandelt, und wie er zu seiner Sichtweise auf die Welt des Wissens und Denkens kam. Auch seine endgültige Antwort, ob es dieses Wissen, außerhalb eigener Erfahrungen geben kann, wurde ausführlich behandelt und erklärt. Da ich mich privat schon länger mit Kant und seinen verschiedenen Werken befasst habe, war ich begeistert davon, genau dazu einen Vortrag eines hervorragenden Professors im Bereich der Philosophie in Aalen an der Hochschule besuchen zu können.\vspace{0.5cm}
\item\textbf{Cybersicherheit in der Digitalisierung}\par

\vspace{0.5cm}
\item\textbf{Ethik Cafe Zeug über Mensch}\par

\vspace{0.5cm}
\item\textbf{Ethik: Was soll ich tun?}\par

\vspace{0.5cm}
\item\textbf{Safe performance with Total Care Tools}\par

\vspace{0.5cm}
\item\textbf{Live-Hacking: So brechen digitale Angreifer in ihre Systeme ein}\par

\vspace{0.5cm}
\item\textbf{Sinn: Was darf ich hoffen?}\par
\end{enumerate}
\clearpage
\section{Beschreibung der sozialen Tätigkeit}
\clearpage
\section{Eidesstattliche Erklärung}
\vspace{1cm}
Ich erkläre hiermit, dass ich die angegebenen Veranstaltungen des Studium Generale persönlich
besucht habe und die vorliegenden Angaben zu den von mir besuchten Veranstaltungen
und sozialen Tätigkeiten wahrheitsgetreu und selbständig verfasst habe.
Weiterhin versichere ich, keine anderen als die angegebenen Quellen und Hilfsmittel benutzt zu haben, dass alle Ausführungen, die anderen Schriften wörtlich oder sinngemäß entnommen
wurden, kenntlich gemacht sind und dass die Arbeit in gleicher oder ähnlicher Fassung noch
nicht Bestandteil einer Studien- oder Prüfungsleistung war.
\vspace{2cm}\\
\begin{tabular}{lp{2em}l} 
 \hspace{5cm}   && \hspace{5cm} \\\cline{1-1}\cline{3-3} 
 Ort, Datum     && Unterschrift(Student) 
\end{tabular}

\clearpage
\section{Anlagen}

\end{document}